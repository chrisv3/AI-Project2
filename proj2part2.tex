\documentclass[a4paper]{article}

\usepackage[english]{babel}
\usepackage[utf8x]{inputenc}
\usepackage{amsmath}
\usepackage{graphicx}
\usepackage[colorinlistoftodos]{todonotes}

\title{CMSC 471: Artificial Intelligence}
\author{Christine Vu}

\begin{document}
\maketitle

\section{}
When applying the function
\[z = \frac{\sin(x^2 + 3y^2)}{0.1 + r^2} + (x^2 + 5y^2) * \frac{e^{1-r^2}}{2}, r = \sqrt{x^2 + y^2}  \]
to hill-climbing, hill-climbing with random restarts, and simulated annealing, hill-climbing was the shortest. I think this is so because out of the other algorithms, hill-climbing is the simplest. Once it finds a minimum, the algorithm will stop, unlike hill-climbing with random restarts and simulated-annealing. Overall, however, simulated annealing produced the most accurate output. This may be because of the opportunities of exploration simulated-annealing are provided with. With the acceptance probability, there is more risk taking in finding a potential global minimum, instead of settling. 

To find visual representations of these functions applied to these algorithms, run 'graphs.py.' No other command arguments are needed. The points a a little difficult to see, but if you rotate the image, at the right angle you can see them. They look like little red triangles. 




\end{document}
